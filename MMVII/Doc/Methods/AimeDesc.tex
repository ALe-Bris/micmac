
\chapter{Description de la m\'ethode Aim\'e}


\section{Introduction}

Aim\'e vise \`a \^etre une m\'ethode de calcul de points homologues qui sera  int\'egr\'ee dans MMVII. 
Son architecture est fortement inspir\'ee par SIFT qui a fait ses preuves depuis de longues
ann\'ees en tant que m\'ethode \emph{analytique}\footnote{"analytique" par opposition \`a  m\'ethode d'apprentissage,
les ayatolah du deep dirait handcrafted} de r\'ef\'erence. Elle vise aussi \`a tirer parti de $10$ ann\'ees
d'exp\'erience d'utilisation de SIFT en photogramm\'etrie pour tirer parti de ses points forts
et corriger les principaux point faibles
de SIFT dans ce contexte.  Notemment les principaux points  sur lesquels on souhaite am\'eliorer sont :


\begin{itemize}
   \item SIFT passe difficilement \`a l'\'echelle lorsque  l'on l'utilise sur des tr\`es grands jeux de donn\'ees,
         notamment il ne permet pas de d\'etecter rapidement les paires potentiellement homologues;

   \item lorsque l'on en dispose, SIFT n'utilise pas d'information  de spatialisation approch\'ee qui permetrait
         de faciliter l'appariement (plus robuste et plus rapide), voir de se passer dans ce contexte d'invariance
         inutile (i.e ne pas normaliser \`a l'\'echelle ou \`a la rotation si on a des information permettant de
         connaitres ces infos);

   \item tirer parti des m\'ethodes d'apprentissage pour avoir un appariement meilleurs (quoique veuille dire
         ceci); une contrainte pour tirer parti des m\'ethodes d'apprentissage est de disposer de  jeux de
         donn\'es de v\'erit\'e suffisemment denses; or cela est possible, au moins pour la phase d'appariement 
         en utilisant des jeux de donn\'ees trait\'es par des cha\^ines photogramm\'etriques automatiques.

   \item prendre en compte que les probl\`emes d'appariement rencontr\'es en photogramm\'etrie sont
         suffisement vari\'es pour qu'il soit n\'ecessaire de viser \'a une m\'ethode fortement param\'etrable
         pour pouvoir s'adapter \`a cette vari\'et\'e de probl\`emes.


\end{itemize}

    % ==================================================================================
    % ==================================================================================
    % ==================================================================================

\section{Architecture g\'en\'erale d'Aim\'e}

L'architecture est la suivante :

\begin{itemize}
   \item calcul de points caract\'eristiques, cette partie est celle qui resssemble le plus \'a SIFT;

   \item calcul de descripteurs, le descripteur est fait d'un (ou plusieurs) descripteurs binaires
         permettant de faire une pr\'e-s\'election rapide des couples potentiellement appariables et
         d'un descripteurs plus complets , ces descripteurs sont bas\'es sur un r\'e\'echantillonage
         log-polaire de du voisinage de chaque point;

   \item appariement en plusieurs \'etapes, on commence par des filtres relativement peu s\'electifs
        mais rapides et l'on compl\`ete par des calcul plus s\'electif sur les descripteurs complets;

   \item filtrage spatial imposant une coh\'erence spatiale sur le principe "les homologues de mes voisins
         sont les voisins de mes homologues"; \'eventuellement la phase d'appariement a pu \^etre volontairement
         ambigu\"e (chaque point a un ensemble d'homologues potentiels) , parce que  certaines ambigu\"it\'es ne
         peuvent  pas \^etres r\'esolues au niveau individuel, et la phase de filtrage cherchera \`a r\'esoudre
         ces ambigu\"it\'es par une approche de type relaxtion.

   \item \'eventuellement, am\'elioration de la pr\'ecision de localisation par des m\'ethodes basiques 
         de corr\'elation.

\end{itemize}

Le degr\'e d'impl\'ementation actuel est tr\`es in\'egal et diminue lorsque l'on avance dans le processus.
Sch\'ematiquement ;

\begin{itemize}
   \item pour les points caract\'eristiques, il existe un prototype complet que l'on peut consid\'erer comme 
         pr\'e-op\'erationnel; il est surement perfectible \'a terme, mais ce n'est pas une priorit\'e; 

   \item le descripteur en log-polaire est implant\'es, des descripteurs binaires ont \'et\'e prototyp\'es de
         mani\`ers handcrafted (par une ACP),  mais gagneraient problement \'a \^etre con\c{c}u avec une 
         approche bas\'ee sur de l'apprentissage;

   \item l'appariement a \'et\'e impl\'emnt\'e de mani\`ere basique, essentiellement pour valider les deux
         \'etapes pr\'c\'dentes;  une voie d'am\'elioration majeure (en tout cas esp\'er\'ee comme telle)
         serait de faire de l'apprentissage sur les paires de descripteur complets;

   \item rien n'a \'et\'e fait sur le filtrage spatial, il aussi esp\'er\'e  que l'approche 
         "appariement ambigue/filtrage spatial avec relaxation" soit une source d'am\'elioration importante
         dans certain cas , notamment le cas de sc\`enes avec des structures partiellement r\'ep\'etitives;

   \item enfin le raffinement par corr\'elation, least square matching n'a pas \'et\'e test\'e, il s'agit
         d'une option a priori facile, qui donnerait une plus value op\'erationnelle sans avoir de valorisation
         en recherche (c'est assez classique).

\end{itemize}


    % ==================================================================================
    % ==================================================================================
    % ==================================================================================

\section{Calcul des points caract\'eristiques}


Ce qui est  d\'ecrit dans cette section est compl\`etement r\'ealis\'e dans  {\tt MMVII}
par la commande {\tt TieP-AimePCar}. Notons qu'\`a plusieurs endroit cette commande
utilise les service de {\tt MMV1}.

    % ==================================================================================

\subsection{Multi-\'echelle}

\subsubsection{Pr\'esentation g\'en\'erale}

Comme indiqu\'e pr\'ec\'edemment, cette partie est celle qui s'inspire le plus directement de SIFT.
Notamment, comme dans SIFT :

\begin{itemize}
    \item on calcule une pyramide d'image d'echelle d\'ecroissante suivant une loi exponentielle, 
          on note $I_0$ l'image initiale, et $I_k$ la $k^{ieme}$ image, entre $I_k$ et $I_{k+1}$
          il y a un rapport d'\`echelle $\sigma$, donc par it\'erations successive entre $I_0$
          et $I_k$ il y a un rapport d'\'echelle $\sigma^k$;

    \item comme dans SIFT, $\sigma$ est choisi tel que $\sigma^n=2$ ou $n$ est un entier, et
          tous les $n$ l'image est d\'ecim\'e d'un facteur $2$ pour gagner un facteur important de temps
          de calcul, sachant que quand l'image est devenu suffisement floue, l'auto-corr\'elation entre un
          pixel et son voisin fait que l'on perd peu d'information avec cette d\'ecimation;

    \item  la pyramide d'\'echelle permet d'avoir plus de point quand cela est n\'ecessaire, et surtout
           permet d'obtenir l'invariance \`a l'\'echelle (pour deux images $I$ et $J$ prises \`a deux
           \'echelles diff\'erentes de rapport $R$, si un point $p$ est d\'etect\'e sur $I_k$, il pourra aussi
           \^etre d\'etect\'e sur $J_{k'}$ avec $R = \sigma^{k-k'}$ )

    \item  la convolution par une gaussienne est isotrope ce qui  est n\'ecessaire pour l'invariance par rotation;


\end{itemize}

Le fichier {\tt ImagesFiltrLinear/cGaussPyram.cpp} contient l'impl\'ementation du calcul de la pyramide.


\subsubsection{Imp\'ementation dans MMVII de la pyramide}

\emph{Il n'est pas prouv\'e que les choix d'impl\'ementation d\'ecrit ici soit judicieux, cette description
vise \`a comprendre le code. Des tests de performance \'a une impl\'ementation basique reste \`a effectuer.}

L'impl\'ementation repose sur deux remarques :

\begin{itemize}
    \item en vertu du th\'eor\`eme central limite , si on convolue avec suffisement de fois avec des fonctions r\'eguli\`eres,
          on converge vers la convolution avec une gaussienne;
    \item convoluer avec deux gaussiennes sucessive d'\'ecart type $a$ et $b$, a exactement le même effet que
          de convoluer avec une seule gaussienne d'\'ecart type $\sqrt{a^2+b^2}$;
  
\end{itemize}

Le calcul se fait en iterant une convolution par $e^{-a|x|}$ qui est une fonction 
"smooth" et dont le produit de convolution peut \^etre calcul\'e rapidement (algorithme r\'ecursif classique).
La fonction est impl\'ement\'ee dans {\tt ExpFilterOfStdDev}.  Cette fonction prend
en param\`etre un $\sigma$ et un nombre d'it\'eration $N$, elle calcule la valeur
$a$ qui donnera un \'ecart type de $\frac{\sigma}{\sqrt{N}}$,

A chaque octave  :

\begin{itemize}
   \item  pour $I_0$ le calcule se fait avec un nombre relativement important d'it\'eration pour
          bien approximer la gaussienne

   \item  ensuite le calcul de $I_{k+1}$ se fait \`a partir $I_k$ (en tenant compte du flou d\'ej\'a pr\'esent
          en $I_k$, on convolue par $\sigma^k \sqrt{\sigma^2-1}$), comme $I_k$ est d\'ej\'a le r\'esultat de plusieurs
          convolution, on peut faire moins d'it\'eration (MMVII en fait $4$ pour $I_0$, $3$ pour $I_1$, et $2$ ensuite);
\end{itemize}

Il y a un cas particulier pour la premi\`ere it\'eration, si on veut que la pyramide soit r\'eguli\`ere en \'echelle.
Typiquement si l'image initiale est tr\`e floue, il faudrait la d\'econvoluer, inversement si elle est tr\`es piqu\'ee,
il faudra la flouter.  Je ne vois pas d'autre solution que de faire une hypoth\`ese sur la largeur de la t\^ache image.
Voir chapitre suivant pour voir comment le calcul de la convolutin sur $I_0$ est g\'er\'e.

    % ==================================================================================

\subsection{Extraction de points caract\'eristiques}

        %  -  -  -  -  -  -  -  -  -  -  -  -  -  -  -  -  -  -  -  -  -  -  -  -  -

\subsubsection{Approche multi-crit\`ere }

L'approche dans Aim\'e est d'avoir potentiellement une grande vari\'et\'e de point caract\'eristique,
pour permettre \`a l'utilisateur de choisir un sous-ensemble en fonction des caract\'eristiques
sp\'ecifiques de son acquisition.

Typiquement :

\begin{itemize}
   \item sur un chantier Satellitaire (ou a\'erien avec GPS/INS) , avec 
         des tr\`es grandes images et une mise en place approch\'ee,
         le nombre de points n'est pas dimensionnant et on pourra de limiter \'a un seul type (par exe
         les maximum du laplacien) pour acc\'elerer tous les calculs;

   \item sur un chantier peu structur\'e ayant du mal \`a se mettre en place, on pourra choisir de
         calculer tout les types de points;
\end{itemize}

Actuellement, il y a $3$ type de points calculables, chacun pouvant être d\'eclin\'e en maxima
ou miminima, ce qui donne $6$ \'etiquettes possibles:

\begin{itemize}
   \item les points de type laplaciens, qui sont comme dans SIFT calcul\'es par  une diff\'erence de
         gaussienne entre niveaux successifs;

   \item les points qui sont simplement un minima ou maxima du niveau de gris; l'id\'ee est que pour des
         image pauvres en contraste et en r\'esolution (par ex images thermiques), le mieux est d'avoir le
         minimum de filtre et qu`\^etre un maxima/minima (point chaud/point froid) est d\'ej\`a un invariant;


   \item les points de coins qui sont calcul\'es par application d'un filtre qui impl\'emente
          la notion de courbure de la courbe de niveau; typiquement il s'agit de calculer la valeur
          de la hessienne dans la direction orthogonale au gradient; l'impl\'ementation actuelle
          utilise la fonction {\tt courb\_tgt} de MMV1;
\end{itemize}


Le choix de ces $3$ crit\`eres est tr\`es arbirtraire et sans doute provisoire. Le principal 
\`a court terme \'etait d'avoir une architecture de code multi-crit\`ere permettant d'\'evoluer.


        %  -  -  -  -  -  -  -  -  -  -  -  -  -  -  -  -  -  -  -  -  -  -  -  -  -

\subsubsection{Extraction des extrema}

Le principe d'extraction est \`a nouveau calcqu\'es sur SIFT, on s\'electionne pour chaque
crit\`ere les point qui sont maxima (vs minima) \`a la fois en spatial et en \'echelle.
Quelques diff\'erences mineures par rapport \`a SIFT :

\begin{itemize}
  \item pour limiter le nombre de point extraits en amont, les maxima spatiaux sont 
        extraits sur voisinage plus grands que $1$ pixel (valeur param\'etrable)

  \item l'affinage sub-pixellaire est fait en espace mais pas en \'echelle;  je n'ai pas
        de justification \'evidente de ce choix sinon que : c'est plus simple et surtout 
        si on a des point compris entre deux \'echelle on est en permanence en train d'aller
        interpoller entre deux \'echelle pour lire les niveau de gris;

\end{itemize}


Le code correspondant \`a ces calcul se trouve dans le fichier {\tt ImagesInfoExtract/ExtracExtremum.cpp} :


\begin{itemize}
   \item  la fonction {\tt ExtractExtremum3} effectue ce calcul en prenant en argument $3$ image
          (l'image elle m\^eme {\tt ImC} et une image du dessus et du dessous), un rayon et une 
          structure pour enregistrer les r\'esultat;


   \item le calcul est fait dans {\tt TestIsExtre3}, le principe est de faire tr\`es vite des tests
         basiques qui r\'efuteront la plupart des candidats potentiels puis de faire des test plus
         long avec des boucles de voisinage ;

    \item le code est fait pour traiter rigoureusement les plateaux (zones connexes de points ayant
          exactement la même valeur), pour avoir exactement un seul maximum sur ces plateau la r\`egle
          qui est utilis\`e pour comparer deux pixels est de comparer par priorit\'e : leur valeur image,
          puis leur \`echelle, puis leur $x$, puis leur $y$; ceci est r\'ealis\'e par les fonction
          {\tt IsImCSupCurP} pour les points de m\^eme \`echelle et {\tt IsImUpSupCurP} et {\tt IsImBotSupCurP}
          pour ceux d'\'echelles diff\'erentes.

\end{itemize}

        %  -  -  -  -  -  -  -  -  -  -  -  -  -  -  -  -  -  -  -  -  -  -  -  -  -

\subsubsection{Adaptation de la dynamique \`a l\'echelle}

Si on consid\`ere les points qui sont des maximas/minima du niveau de gris,
on voit que comparer brutalement les points avec l'echelle du dessous n'a pas
de sens car plus on convolue plus applatit les extrema et un extrema ne sera
jamais sup\'erieur \`a celui qui le pr\'ec\`ede en \'echelle. Pour que la comparaison
soit \'equitable il faut donc adapater la dynamique en fonction de l'\'echelle.

Intuitivement on serait tent\'e dans ce cas de simplement multiplier l'image 
par  $\sigma$, mais cela n'est pas suffisant car si l'on ajoute une constante \`a l'image
on voit que la multiplication par $\sigma$ va \`a nouveau biaiser le r\'esultat. L'id\'ee
est que la multiplication doit se faire sur des images de moyenne nulle, on soustrait donc \`a
l'image une image de moyenne locale.  Pour les autres filtres utilis\'es comme il sont
diff\'rentiel (laplacien, coins) il n'y a pas de probl\`eme de moyenne.


Les compensation de dynamique sont alors les suivantes :

\begin{itemize}
   \item multiplication par $\sigma^3$ pour les coins, (voir {\tt cGP\_OneImage<Type>::MakeCorner()})

   \item multiplication par $\sigma$ pour les l'image initiale centr\'ee (void {\tt cGP\_OneImage<Type>::MakeOrigNorm()})

   \item rien pour l'image de laplacien (parce que la diff\'erence de gaussienne n'est pas vraiment un laplacien
         mais d\'ej\`a un laplacien multipli\'e par $\sigma^2$, voir {\tt cGP\_OneImage<Type>::MakeDiff()})

\end{itemize}
 
    % ==================================================================================

\subsection{Filtrage a priori}

Toute la difficult\'e est d'avoir suffisement de point dans les cas n\'ecessaire et pas trop
dans les autres \dots

        %  -  -  -  -  -  -  -  -  -  -  -  -  -  -  -  -  -  -  -  -  -  -  -  -  -

\subsubsection{Filtrage par auto corr\'elation}

Supprime les point qui auto-corr\`elent avec leur voisinage.

\subsubsection{Filtrage spatiale \emph{a priori}}

Impose une certaine homog\'en\'eit\'e de repr\'esentation, en supprimant les point 
qui des voisin trop proche et "meilleurs" qu'eux. Suppose une \'evaluation a priori
de la qualit\'e. Celle-ci m\'elange : 

\begin{itemize}
   \item l'auto corr\'elation ;
   \item l'\'echelle (a priori les point situ\'e dans les basses \'echelles sont plus stables,
         et comme ils sont plus rares, le risque est faible \`a les privil\'egier);
   \item  la variance locale de l'image;
\end{itemize}
 

    % ==================================================================================
    % ==================================================================================
    % ==================================================================================

\section{Descripteur}

    % ==================================================================================

\subsection{Vignette log-polaire}



Le calcul des descripteur est enti\'erement bas\'e sur un r\'eechantillonage log-polaire 
au voisinage du point.  La justification th\'eorique est assez similaire \`a celle du descripteur
daisy ({\tt google daisy descriptor}).  L'int\'er\^et attendu d'un  r\'eechantillonage log-polaire
est le suivant concernant l'aspect polaire:

\begin{itemize}
   \item le  r\'eechantillonage polaire permet de convertir les rotation en translation donc
        d'effectuer de mani\`ere plus efficace les op\'eration de rotation si elles doivent \^etre
        nombreuses;
   \item dans le cadre d'une d\'etection invariante \`a la rotation, le  r\'eechantillonage polaire
        est isotrope (ne privil\'egie aucune direction particuli\`ere) contrairement \`a une vignette en $x,y$.
\end{itemize}

C'est surtout la combinaison avec l'aspect  log qui est int\'eressante vis  \`a vis des crit\`ere suivants:

\begin{itemize}
   \item on souhaite avoir une vision globale permettant d'avoir une information sur le "contexte"  du 
         point, en concervant une taille raisonnable de  vignette;

   \item compte tenu des effets de d\'eformation perspectives (dues au relief et \`a l'impr\'ecision sur la
         normalisation par rotation), plus on s'\'eloigne du centre, plus il faut diminuer la pas d\'echantillonnage;

   \item typiquement, le pas d'\'echantillonage doit \^etre  proportionnel \`a la distance, ce qui m\`ene
         \`a une loi exponentielle pour l'\'echantillonage en distance;

   \item par ailleurs pour que l'image soit bien \'echantillon\'ee, plus on s'\'eloigne en distance, plus la 
         radiom\'etrie doit \^etre calcul\'ee en moyennant sur un grand voisinage, ce qui se fait naturellement
         en allant lire les radiom\'etrie dans la pyramide d'image;
\end{itemize}


Les consid\'erations pr\'ec\'edentes m\`enent aux formules  ~\ref{NoteLP} et  \ref{DefLP}.
Soit un point $x,y$ extrait dans une image $I_k$, on
va calculer un image $\LP(u,v)$ d\'efinie par :


\begin{equation}
    \theta(u) = \frac{2\pi u}{\omega} \;\; \rho(v)= \rho_0 \sigma^{v*B} \label{NoteLP}
\end{equation}

\begin{equation}
    \delta_x(u,v) =  \rho(v) \cos(\theta(u))  \; \;   \delta_y(u,v) =  \rho(v) \sin(\theta(u))  \label{XYLP}
\end{equation}

\begin{equation}
     \LP(u,v) =   I_{[k+\delta_{k0}+v*B]}(x + \delta_x(u,v), y+\delta_y(u,v)) \label{DefLP}
\end{equation}


Les param\`etre de ce r\'e\'echantillonage, sont control\'es par  argument {\tt LPC} et {\tt LPS}
de la commande {\tt TieP-AimePCar} :

\begin{verbatim}
 * [Name=LPC] std::vector<double> :: Circles of Log Pol [Rho0,DeltaI0,DeltaIm] ,[Default=[2.5,-1,2]]
 * [Name=LPS] std::vector<double> :: Sampling Log Pol [NbTeta,NbRho,Mult,Census] ,[Default=[16,8,32,0]]
\end{verbatim}

Dans~\ref{DefLP} :

\begin{itemize}
   \item  $\omega$ est la fr\'equence angulaire (par d\'efaut $16$), ce param\`etre fixe \`a la fois le
          pas et le nombre de valeurs angulaires;
   \item  $B$ vaut $2$ par d\'efaut 
   \item  $\delta_{k0}$ vaut $-1$ par d\'efaut, cela n'impose pas de restriction sur le calcul car
          par construction des extrema, un point extrait a toujours une image au dessus;
   \item  $\rho_0$ vaut $2.5$;
   \item sinon le nombre d\'echantillonage en $\rho$ est par d\'efaut de $8$, compte tenu des autre valeurs
         par d\'efaut cela correspond \`a \'echantilloner sur $3$ octave ($B$ vaut $2$, donc on a 
         va regarder jusqu`a $15=8*2-1$ images au dessus, et il y a $5$ \'echelles par octaves);
\end{itemize}


    % ==================================================================================

\subsection{Invariance par rotation}

Deux approches ont \'et\'e test\'ees  : une approche par calcul descripteur
invariant \`a la rotation, une approche par calcul de direction principale.
On d\'ecrit les $2$, sachant que pour l'instant la premi\`ere approche est un
peu abandonn\'ee.

On note $\LP^\delta$ la translat\'ee de $\LP$ sur l'axe $u$, correspondant donc \`a une
rotation de l'image :

\begin{equation}
    \LP^\delta(u,v) = \LP(u+\delta,v)
\end{equation}

        %  -  -  -  -  -  -  -  -  -  -  -  -  -  -  -  -  -  -  -  -  -  -  -  -  -

\subsubsection{Calcul d'invariant}

On cherche a d\'efinir des grandeur qui soit invariante \`a une translastion
circulaire de $\LP$. Quelques tentatives bas\'ees ur l'auto-corr\'elation  :

\begin{equation}
    AC(\delta,v) =  \int \LP * \LP^\delta
\end{equation}

\begin{equation}
    AC_u(\delta,v) =  \int |\frac{\partial \LP}{u}| * \LP^\delta
\end{equation}

\begin{equation}
    AC_v(\delta,v) =  \int |\frac{\partial \LP}{v}| * \LP^\delta
\end{equation}

Ces invariants sont implant\'es dans {\tt MMV1} mais pas dans {\tt MMVII}.

        %  -  -  -  -  -  -  -  -  -  -  -  -  -  -  -  -  -  -  -  -  -  -  -  -  -

\subsubsection{Calcul d'une direction principale}

Le principe est de d\'efinir un estimateur de direction $\theta(\LP)$ qui v\'erifie
axiomatiquement :

\begin{equation}
     \theta (\LP^\delta) = \delta + \theta (\LP)
\end{equation}

Si on dispose d'un tel estimateur, on peut d\'efinir une forme invariante $\LP_\theta$ par rotation
par la formule :


\begin{equation}
     \LP_\theta = \LP^{-\theta(\LP)}
\end{equation}

Quelques exemples de tels estimateurs impl\'ement\'es dans  {\tt TieP-AimePCar} :

\begin{itemize}
   \item $\theta^L$  on calcule pour chaque $u$ la somme de $H(u) = \int \LP(u,v) d_v$
         et  $\theta^L = \argmax (H)$

   \item $\theta^u$  idem mais sur un histogramme de gradient en $u$ :  $H_u(u) = \int |\frac{ \partial \LP(u,v)}{\partial u}| d_v$

   \item $\theta^v$  idem mais sur un histogramme de gradient en $v$ :  $H_v(u) = \int |\frac{ \partial \LP(u,v)}{\partial v}| d_v$
\end{itemize}


Ces estimateurs sont implant\'es dans {\tt CalcOrient} du fichier {\tt CalcDescriptPCar/cAimeTieP.cpp}.

        %  -  -  -  -  -  -  -  -  -  -  -  -  -  -  -  -  -  -  -  -  -  -  -  -  -

\subsubsection{Choix d'une approche par directions principales multiples}

Les m\'ethodes par invariants, de type auto-corr\'elation, on \'et\'e intoduites dans {\tt MMV1}
pour contourner l'inconv\'enient suivant , quelques soit l'estimateur de direction choisi, il existe des formes
pour lequel celui est ind\'efini, typiquement il peut y avoir plusieurs $u$ r\'ealisant le max et $\argmax$
est ambigu, et proche de ces situation ambigues, l'estimation est instable.
Par ailleurs, ceci n'est pas du \`a un mauvai choix de l'estimateur, on peut montrer que c'est intrins\`eque
au probl\`eme.


Cependant les m\'ethodes par invariant pr\'esente d'autre inconv\'enients :

\begin{itemize}
   \item il est difficile de savoir si il sont complets au sens o\'u l'on voudrait \^etre certain
         que deux formes auront le  m\^eme ensemble d'invariant si et seulement si elle sont identiques
         \`a une rotation pr\`es;

   \item ailleurs la m\'etrique sur ces invariants est assez \'eloign\'ee de celle sur la forme initiale;

   \item  ils sont relativement  volumineux en m\'emoire, et comme ils sont incomplets ce co\^ut vient
          s'additionner  \`a celui de $\LP$ qui  doit \^etre aussi sauvegarder.
\end{itemize}


Une solution qui semble n'avoir aucun des $2$ inconv\'enient est la suivante :

\begin{itemize}
   \item on se donne plusieurs estimateurs $\theta_0$, $\theta_1$ , $\theta_2$ \dots 

   \item les pr\'e appariement rapides li\'es aux index binaires (voir \RefFantome)  sont
         fait sur les diff\'erents estimateurs, et ensuite les appariement plus fin sont
         fait uniquement sur le meilleur pr\'e-appariment rapide (dans l'hypoth\`ese o\`u
         deux points sont pr\'e-appari\'es  plusieurs fois);

   \item  un compl\'ement \`a cette approche, prometteuse mais non encore impl\'ement\'ee, est de se donner
          pour chaque estimateur  direction un estimateur de sa stablilit\`e , c'est assez facile
         \`a d\'efinir avec les estimateurs bas\'es sur les histogrammes; 
          par exemple si  le maximum est r\'ealise en $v_m$ une formule bas\'ee sur la d\'eriv\'e logarithique
          tel~\ref{DerLogH}
% est  second x situ\'es en $v_1$ et $v_2$,
          % on peut envisager $\frac{M_1+M_2}{M_1-M_2}
\end{itemize}

\begin{equation}
    \min_{v!=v_m}\frac{|H(v_m)-H(v)|}{H(v_m)*|v_m-v|} \label{DerLogH}
\end{equation}

C'est a priori vers cette solution que l'on se dirige : multi estimateur, avec \'eventuellement estimateur
de stabilit\'e. Notons que le fait d'avoir ces estimateurs de stabilit\'e ne dispense pas d'avoir
plusieurs estimateurs d'orientation: retenir uniquement celui qui est le plus stable serait une fausse bonne id\'ee
car dans les cas o\`u les deux estimateurs de stabilit\'es ont une valeur proche on sera \`a 
nouveau dans une situation instable.

    % ==================================================================================
    % ==================================================================================
    % ==================================================================================

\section{Descripteurs binaires}

Si le module de calcul des points caract\'eristique et des descripteur $\LP$ existe sous 
la forme d'un prototype complet et maintenu sous {\tt MMVII}, les id\'ees d\'ecrites soient sont
purement th\'eorique (notamment tout ce qui concerne l'apprentissage profond) soit ont
\'et\'e \'ebauch\'ees dans des prototype qui ne sont plus maintenus (notamment les descripteurs
binaires).

    % ==================================================================================

\subsection{D\'efinitions}

        %  -  -  -  -  -  -  -  -  -  -  -  -  -  -  -  -  -  -  -  -  -  -  -  -  -
\subsubsection{Descripteur \`a un bits}
Un desccripetur binaire/de bits est simplement une fonction qui associe \`a une forme une valeur bool\'eenne.
Dans notre cas le descripteur:

\begin{equation}
    B(\LP) \in \{0,1\}
\end{equation}

        %  -  -  -  -  -  -  -  -  -  -  -  -  -  -  -  -  -  -  -  -  -  -  -  -  -
\subsubsection{"Petits" descripteur \`a N bits}

Classiquement, \`a partir de $n$ descripteur de bits,  on identifie
le vecteur de bits $\ChBit = [B_0, B_1, \dots , B_{n-1}]$ avec les entiers en base $2$
et on \'ecrit   :

\begin{equation}
   \ChBit(\LP)  =  \sum_{k=0}^{k<n} B_k(\LP) 2^k  =  [B_0(\LP), \dots ]
\end{equation}

L'int\'er\^et de la codification enti\`ere est que, pour les  cardinaux assez petit, les op\'eration
peuvent \^etre faite avec un co\^ut ind\'ependant de $n$.  Notamment :

\begin{itemize}
   \item le calcul du cardinal/nombre de bit peut faite en "tabulant" ses valeurs pour tous les entiers $<n$
         dans un tableau $C^{ard}$;

   \item les op\'eration ensembliste d'intersection, union diff\'erence sym\'etrique peuvent être faite gr\^ace aux
         op\'eration "bits \`a bits"  : {\tt \bf  \&,|,\^{}}

   \item par exemple la distance de hamming entre deux champs de bit $\ChBit_1$ et $\ChBit_2$  peut \^etre 
         calcul\'ee par $C^{ard}(\ChBit_1 \hat{~}  \ChBit_2)$

\end{itemize}

On appellera petit descripteur \`a $N$ bits des descripteurs qui sont tabulables de mani\`ere raisonnable,
typiquement ce sera en pratique des descripteur pour lesquels $N$ est compris entre $16$ et $24$.

Ces "petits" descripteurs ne sont en g\'en\'eral pas suffisant pour prendre un d\'ecision d\'efinitive
avec une distance type distance de Hamming, par contre il peuvent \^etre utilis\'e pour faire une pr\'eselection
rapide en supprimant une grande proportion de faux homologues et conservant la grand majorit\'e des vrais.

        %  -  -  -  -  -  -  -  -  -  -  -  -  -  -  -  -  -  -  -  -  -  -  -  -  -
\subsubsection{Grand Descripteur \`a N bits}

Lorsque les descripteur sont trop grands pour que l'on puisse tabuler le nombre de bits,
il est encore possible de tirer parti du parall\'elisme de la fa\c{c}on suivante :

\begin{itemize}
    \item on d\'ecoupe le champ de $\ChBit$ de  $N$ bits en $k=\frac{N}{n}$ champs $\ChBit_1 \dots \ChBit_k$ de
          taille $n$;

     \item pour calculer la distance de Hamming $H$ entre $\ChBit$ et $\ChBit'$ on utilise
           la formule~\ref{DistGrdCh}
\end{itemize}

\begin{equation}
   H(\ChBit,\ChBit') = \sum_k H(\ChBit_k,\ChBit_k') = \sum_k C^{ard}(\ChBit_k \hat{~}  \ChBit_k') \label{DistGrdCh}
\end{equation}

Par exemple avec un champs de $80$ bits on peut calculer des distances de hamming par la somme de $5$
distance de hamming sur des champs de $16$ bits, taille pour laquelle la tabulation de $ C^{ard}$
est tout \`a fait raisonnable.

    % ==================================================================================

\subsection{Utilisation dans MMV1/MMVII}

        %  -  -  -  -  -  -  -  -  -  -  -  -  -  -  -  -  -  -  -  -  -  -  -  -  -
\subsubsection{Architecture}

L'architecture prototyp\'ee dans {\tt MMV1} et a priori envisag\'ee dans {\tt MMVII}
pour faire un appariement rapide est bas\'e sur les principe suivant :

\begin{itemize}
    \item d'abord, et c'est assez trivial, on n'apparie que les points de m\^eme nature
          (par exemple les laplacien-max entre eux), toute les structures (tables d'indices)
          existent en autant d'exemplaires qu'il y a de type de points; 

    \item on fait une premi\`ere pr\'e s\'election sur des  petits descripteurs binaires permettant
          de cr\'eer des structures d'indexation;

    \item on fait une deuxi\`eme pr\'e s\'election sur des  grands descripteurs  plus s\'electifs
          et encore assez rapide;
          
    \item finalement, sur les couple qui restent, on fait un calcul qui "attaque" directement les descripteurs
          $\LP$.
\end{itemize}

Pour donner un peu plus de concret, donnons quelques chiffres invent\'es mais r\'ealistes ;

\begin{itemize}
   \item  on doit apparier deux images contenat chacune $60 000$ points caract\'eristiques ;
   \item  ces points correspondent \`a $6$ \'etiquettes, que l'on suppose par simplicit\'e r\'eparties de
          mani\`eres homog\`enes en bloque de $10000$;

   \item  chaque point a priori $10000$ candidat \`a tester; 
  
   \item  on a  cr\'e\'e pour les points de l'image $1$ , une table index\'ee (voir~\ref{DescIndex}) 
          d'un descrpteur sur $20$ bit,

   \item on peut acc\'eder directement, pour chaque point de l'image $2$ \`a la liste des point situ\'es
         \`a une distance de Hamming de $5$ (par exemple);  cette liste ne contient en moyenne que
         $500$ point avec une probabilit\'e de $90\%$ de contenir le vrai homologue (si il est pr\'esent);
          
   \item en utilisant un descripteur binaire long sur $160$ bits, et en s'electionnant que les points dont
         la distance de hamming est inf\'erieure \`a $30$, on obtient une liste de $50$ candidats 
         avec une probabilit\'e de $85\%$ de contenir le vrai homologue (si \dots);

   \item ensuite sur les $50$ homologues \`a tester en moyenne, on peut envisager des traitements longs
         qui attaquent directement les images $\LP$.

\end{itemize}

        %  -  -  -  -  -  -  -  -  -  -  -  -  -  -  -  -  -  -  -  -  -  -  -  -  -

\subsubsection{Descripteur binaire et gestion des orientations}

Concernant l'invariance par rotation et la gestion avec plusieurs estimateurs de directions
principales, la solution retenues dans {\tt MMVII} est d'avoir pour chaque estimateur une
duplication des descripteur binaires.

Cette solution est retenue car le coût de stockage de ces descripteur reste faible par rapport
\`a $\LP$, et cette solution est prudente par rapport 

        %  -  -  -  -  -  -  -  -  -  -  -  -  -  -  -  -  -  -  -  -  -  -  -  -  -

    % ==================================================================================
\subsection{Descripteurs index\'es}
\label{DescIndex}

\subsubsection{Principe g\'en\'eral}

Une indexation des descripteur permet de retrouver instantan\'ement, parmi
un ensemble de point carac\'estique, ceux dont le decripteur binaire se trouve
\`a une distance de hamming inf\'erieure \`a un seuil donn\'e d'un autre point.

Le principe g\'en\'eral est le suivant :

\begin{itemize}
   \item on consid\`ere un "petit" descripteur sur $N$ bits;
   \item on cr\'ee une table de liste de pointeur sur les points caract\'eristiques,
         de taille $2^N$, 
   \item soit $d$ le seuil en distance de Hamming que l'on souhaite utiliser, et $S_d$
         l'ensemble des entiers sur $N$ bits, ayant au maximum $d$ bits \`a $1$;
   \item pour chaque point caract\'eristque $P$, soit $\ChBit$ le code binaire de $P$,
         on rajoute le pointeur sur $P$ aux indexes $\ChBit \hat s$ pour tous les  $s \in S_d$;
\end{itemize}

        %  -  -  -  -  -  -  -  -  -  -  -  -  -  -  -  -  -  -  -  -  -  -  -  -  -

\subsubsection{Utilisation}

Pour factoriser le co\^ut de construction de la table, l'usage typique est le suivant :
on construit les tables pour une image $I_0$ et ensuite on l'utilise pour toutes les images $I_k$
sur lesquelles on doit effectuer un appariement avec $I_0$.

Concernant les  directions multiples, une fa\c{c}on de l'utiliser est :

\begin{itemize}
    \item on construit autant de table que d'estimateur de direction;
    \item dans la phase de construction, pour chaque point de $I_0$, 
          on prend son estimateur direction correspondant au meilleur estimateur 
          de stabilit\'e  et on le rentre dans la table correspondante;
    \item dans la phase d'utilisation, pour un point $I_k$, on va exlporer
          chaque estimateur de direction avec sa table correspondante;
\end{itemize}

D'autres strat\'egies sont possibles : avoir tous les estimateurs dans $I_0$ et
$I_k$ (couteux, couples pr\'esent plusieurs fois), n'avoir que les estimateurs les plus tables dans $I_0$ et $I_k$
(\'economique mais risque d'omettre des points lorsque plusieurs estimateurs donnent des r\'eponses proches).


Celle ci a l'avantage d'avoir une taille totale
des tables : qui est ind\'ependante du nombre d'estimateur (chaque point est pr\'esent exactement dans
une seule table) , chaque couple est repr\'esent\'e par un seul de ses estimateurs qui est
le plus stable pour un des points, dans les cas limite o\`u plusieurs estimateur de direction
ont quasimment la m\^eme valeur de stabilit\'e



    % ==================================================================================
    % ==================================================================================
    % ==================================================================================
\section{Programme par Apprentissage}


\subsection{G\'en\'eralit\'es}

Les techniques par apprentissages ont montr\'e des performances int\'eressante lorsque 
l'on est capable de leur fournir des jeux d'entrainement suffisemment massifs.
Or dans le contexte des points de liaisons ces donn\'ees massives peuvent venir de trois 
origines :

\begin{itemize}
    \item donn\'ees issue d'aquisition conjointe lidar/photogramm\'etrique, les point lidar
          permettant de g\'en\'erer des homologues semi-denses;  

    \item donn\'ees issue d'acquisition photogramm\'etrique, le mod\`eles $3D$ dense
          permettant  de valider les homologues;

    \item donn\'ees de simulations, r\'ealistes ou non ;
\end{itemize}




    % ==================================================================================
    % ==================================================================================
    % ==================================================================================
\section{S\'election rapides des paires}

Une seule etiquette
Seulement les echelles hautes

    % ==================================================================================
    % ==================================================================================
    % ==================================================================================
\section{Filtrage spatiales}

Notamment relaxation apr\`es appariement multiple.

    % ==================================================================================
    % ==================================================================================
    % ==================================================================================
\section{Affinement g\'eom\'etrique-densification}

    % ==================================================================================
    % ==================================================================================
    % ==================================================================================

\section{Road map propos\'ee}

Il reste beaucoup de chose \`a faire, il s'agit de proposer une strat\'egie/ordonnancement de d\'eveloppement
prenant en compte des crit\`eres diff\'erents et parfois contradictoire :  valorisation
op\'erationnelle dans {\tt MMVII}, valorisation en recherche et publication acad\'emique,
besoin IGN, d\'ependance entre les diff\'erentes \'etapes, degr\'e d'incertitude sur des hypoth\`ese
\`a v\'erifier \dots 





\begin{equation}
     I_k = I_0  \circledast G(\sigma^k)
\end{equation}


\section{VRAC}

\begin{itemize}
   \item indexe binaire
    \item crit\`ere rapide de d\'etection de paires
\end{itemize}



\begin{equation}
     I_k = I_0  \circledast G(\sigma^k)
\end{equation}


\section{VRAC}

\begin{itemize}
   \item indexe binaire
    \item crit\`ere rapide de d\'etection de paires
\end{itemize}

