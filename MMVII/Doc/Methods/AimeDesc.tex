
\chapter{Description de la méthode Aim\'e}

\section{Introduction}

Aim\'e vise à être une m\'éthode de calcul de points homologues qui sera  intégrée dans MMVII. 
Son architecture est fortement inspirée par SIFT qui a fait ses preuves depuis de longues
années en tant que méthode \emph{analytique}\footnote{"analytique" par opposition à méthode d'apprentissage,
les ayatolah du deep dirait handcrafted} de référence. Elle vise aussi à tirer parti de $10$ années
d'expérience d'utilisation de SIFT en photogrammétrie pour corriger les principaux point faibles
de SIFT dans ce contexte.  Notemment les principaux points qui handicapant :


\begin{itemize}
   \item SIFT passe difficilement à l'échelle lorsque  l'on l'utilise sur des très grands jeux de données,
         notamment il ne permet pas de détecter rapidement les paires potentiellement homologues;

   \item lorsque l'on en dispose SIFT n'utilise pas d'information  de spatialisation approchée qui permetrait
         de faciliter l'appariement (plus robuste et plus rapide), voir de se passer dans ce contexte d'invaroant
\end{itemize}


\section{VRAC}

\begin{itemize}
   \item indexe binaire
    \item critère rapide de détection de paires
\end{itemize}

