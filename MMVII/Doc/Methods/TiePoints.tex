

\chapter{The "Aime" methods for Tie Points computation}


% Conclusion, on peut sans doute limiter le nombre de point avec ScaleStab
% pour filtrage a priori => genre les 500 les plus stable

%---------------------------------------------
%---------------------------------------------
%---------------------------------------------

\section{Fast recognitation}

%---------------------------------------------

\subsection{Motivation}
For each image, we have computed tie points. A tie points is made
of a vector $V \in \RR^n)$ . Typically $V$ is invariant
to the main geometric deformation .  Le  $V_1$ and $V_2$
be two tie points, we note :

\begin{itemize}
   \item  $H_{om}(V_1,V_2) $ the fact that two tie points are homologous;
\end{itemize}

Given $V_1,V_2$, there is of course no no oracle that can indicate if  $H_{om}(V_1,V_2)$,
and classically we want to compute  a fast mathematicall function $\Psi $ that indicate if two vector $V_1$ and $V_2$
correspond to the same tie points .  The ideal function would be such :

\begin{itemize}
   \item  $\Psi(V_1,V_2)  \iff H_{om}(V_1,V_2)$
\end{itemize}


Of course this impossible, and we introduce the miss rate  and fall out:

\begin{itemize}
   \item   miss rate , probability of $\Psi=0$ knowing $H_{om}=1$ , we note $p_m$;
   \item   fall out , probability of $\Psi=1$ knowing $H_{om}=0$, we note $p_f$;
\end{itemize}

As we cannot have the ideal function $\Psi $ such as $p_m=0$ and $p_f=0$,
we have to compromise, and depending on the circunstances, the price of the
two error, will not be the same. Typically, in indexation step,  we are especially
interested to have a low $p_f$; converselly in recognition step we are
especially interested to have a low  $p_m$.


\subsection{Bits vector}


\section{Gaussian pyramid}

\subsection{Computing $\sigma_0$}

\label{GP:SIGMA0}

The gaussian pyramid is made from a succession of image at different scale that
result from a gaussian filter. How can we justify it :

\begin {itemize} 
   \item  image $I_k$ must be at resolution  $R_k =  R_0 s ^k$,
   \item if we assimmilat  $I_0$ to a (sum of) gaussian of std dev $\sigma_0$  , $I_k$
         must be (sum of)  gaussian  $\sigma_0 * R_k$ 
   \item so we can write $I_k = I_0 \ast G(\sigma_k)$ with $\sigma_k^2 + \sigma_0^2 = (\sigma_0  R_k)^2$;
\end {itemize} 

To compute the pyramid, we need an estimation of $\sigma_0$. Which is quite natural, if the initial
image is very blured, it a high $R_0$, and the value $R_1-R_0 = (s-1)R_0$ is also high, which require
a high value for gaussian filter.

We need a way to estimate the initial value $\sigma_0$. Also it's quite arbirtrary, the way it is
done in MMVII is :


\begin {itemize} 
   \item  assimilate $I_0$ to a gaussian of  std dev $\sigma_0$;
   \item  suppose $I_0$ is well sampled (nor blurred nor aliased);
   \item we traduce it mathematically by  :
\end {itemize} 

\begin{equation}
    \int_{-\infty}^{+\infty} I_0 |x| = \frac{1}{2}
\end{equation}

Due to symetry , we can replace by integral on $[0,+\infty]$ and supresse absolute value :

\begin{equation}
    \int_{0}^{+\infty} \frac{x}{\sigma_0 \sqrt{2\pi}} e^{-\frac{x^2}{2\sigma_0^2}}  = \frac{1}{4}
\end{equation}

We integrate :

\begin{equation}
      \lbrack  \frac{-\sigma_0}{\sqrt{2\pi}} e^{-\frac{x^2}{2\sigma_0^2}}\rbrack  _{0}^{+\infty}  = \frac{1}{4}
\end{equation}

So :

\begin{equation}
      \sigma_0 = \sqrt{\frac{\pi}{8}} \simeq 0.626
\end{equation}


%----------------------------------------
%  A conserver : equations TIPE Lolo
%-----------------------------------------
\COM{
\begin{equation}
   \delta R = \frac{1}{\sigma} \frac{L}{S} = \frac{1}{\sigma}  \frac{2 \pi a }{ e \; dz} 
\end{equation}

\begin{equation}
   \phi  = \iint \overrightarrow{B} \overrightarrow{dS} 
         = B_M \cos(\omega t) \pi a^2
\end{equation}

\begin{equation}
   e = -\frac{d\phi}{dt}= B_M \omega  \sin(\omega t)  \pi a^2
\end{equation}


\begin{equation}
   d P_J = \frac{e^2}{\delta R}  
         = \frac{ (B_M \omega \pi a^2  \sin(\omega t))^2 \sigma_e dz }{2 \pi a}
\end{equation}


\begin{equation}
   <d P_J> = \frac{B_M^2 \; \omega ^2 \; \pi \; a^3 \; e \; \sigma \; dz}{4}
\end{equation}

\begin{equation}
  <P_J> =  \int_{z=0}^H <d P_J> = \frac{B_M^2 \; \omega ^2  \; \pi \;  a^3  \; e  \; \sigma  \; H}{4}
\end{equation}

\begin{equation}
  dU = \delta ^2 Q_{creee} +  \delta ^2 Q_{recue de l'air}
\end{equation}

\begin{equation}
  C dT = <P_J> dt - g 2\pi aH(T-T_a)
\end{equation}

\begin{equation}
  C = \mu c 2 \pi a e H
\end{equation}


\begin{equation}
  \mu c 2 \pi a e H \frac{dT}{dt} = \frac{B_M \omega^2 \pi a^3 e \sigma H}{4} - g 2 \pi a H (T-T_a)
\end{equation}


\begin{equation}
  \tau  \frac{dT}{dt} + T = T_{\infty}
\end{equation}


\begin{equation}
  \tau  = \frac{\mu  c e}{g}
\end{equation}

\begin{equation}
  T_{\infty} - T_a = \frac{(B_M \omega a)^2 e \sigma}{8g}
\end{equation}
}



